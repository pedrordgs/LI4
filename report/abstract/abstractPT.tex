\par O presente relatório descreve as três fases do desenvolvimento da aplicação mobile "PORTOURGAL", elaborada no âmbito da unidade curricular de Laboratórios de Informática IV, cujo o seu objetivo é facilitar a atividade turística em Portugal.
\par A primeira fase (Fundamentação) consistiu na apresentação do domínio da aplicação, isto é, em que consistiria a aplicação a que nos estariamos a propor desenvolver. Para isso, foram descritos os objetivos, a contextualização, apresentação do caso de estudo, ferramentar a serem utilizadas, assim como a planificação do projeto e os mockups iniciais da aplicação.
\par A segunda fase (Especificação) corresponde à especificação dos requisitos, onde para além do levantamento e análise de requisitos, foram também criados, com recurso a UML, os modelos de sistema.
Estes modelos correspondem ao diagrama de Use Cases, modelo de domínio e diagrama de classes. Após a conclusão do levantamento de requisitos e a construção dos modelos, procedeu-se então à idealização e elaboração da base de dados.
\par Na terceira e última fase deste projeto (Implementação), são apresentadas as principais funcionalidades implementadas. Sendo esta a última fase do projeto, estamos perante a culminação de todas as fases previamente desenvolvidas, dando origem ao produto final.