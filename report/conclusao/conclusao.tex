Ao longo da realização deste trabalho prático, o grupo foi-se deparando com as diversas fases na produção de aplicações, esforçando-se para ultrapassar todas as dificuldades que delas advêm.

Para a primeira fase – a \textbf{Fundamentação} – foram reunidos os pilares fundamentais do projeto: Contextualizou-se o problema, definiram-se os objetivos a cumprir, estudou-se a viabilidade, complexidade e utilidade do sistema a desenvolver, e de seguida listamos os recursos que serão necessários para as subsequentes fases. Esta fase é fulcral pois catalisam desde já a criação de um caminho para a compreensão e modelação com sucesso do sistema de Software.

O plano de desenvolvimento, que foi apresentado na forma de um Diagram de Gantt, foi uma ajuda preciosa pois permitiu organizar e quantificar o número de horas de trabalho necessárias para o desenvolvimento da peça de software proposta.

Na segunda fase – a \textbf{Especificação} – procedeu-se à modelação do projeto assente nos princípios ajustados na fase anterior. Primeiramente foram identificados os requisitos não funcionais e funcionais. De seguida, elaboraram-se alguns diagramas UML considerados relevantes para a compreensão do sistema a ser desenvolvido, tais como: 
\begin{itemize}
    \item Modelo de Domínio
    \item Diagrama Use Cases
    \item Diagrama de Classes
\end{itemize}

O Modelo de Domínio é claro e conciso. Esta propriedade, assim como a identificação correta dos diferentes papéis dos relacionamentos entre as entidades, vão de encontro aos objetivos que traçamos no começo da concessão do projeto. Relativamente ao Diagrama de Use Cases, permitiu capturar requisitos funcionais, fornecendo assim uma notação diagramática que permite modelar o contexto geral do sistema. O Diagrama de Classes permite ilustrar as classes e o relacionamento entre as mesmas, daí ser tão importante implementar.

Para lá da simplificação do problema, adicionalmente, estes modelos, contribuíram para obtermos uma ideia final clara e objetiva de todo o sistema de software implementado.

Estruturou-se a base de dados sólida e esboçaram-se mockups, idealizando a interface e camadas de apresentação do sistema.

Para a terceira e última fase deste projeto – a \textbf{Implementação} –, debruçámo-nos sobre a sobre a produção do software. Tendo por base todo o trabalho desenvolvido referente à modelação, criou-se uma aplicação Web que considerámos cumprir as diretrizes mais relevantes estabelecidas pelo cliente.

O grupo deparou-se com uma linguagem de programação que não tinha sido trabalhada até ao momento, ferramentas de produção de software desconhecidas e um paradigma de desenvolvimento novo. Todos estes fatores contribuíram para que o planeamento não fosse cumprido como tinha sido idealizado na primeira fase de projeto, levando a que este fosse ocasionalmente alterado. Apesar do grupo perceber que deveria ter seguido à risca o plano idealizado no início do projeto (o que não foi possível por não termos experiência com este tipo de aplicações e linguagem), também soube ter um lado mais prático e pro-ativo, por forma a ter um produto final (inacabado) para entregar ao cliente.

Assim sendo, – e apesar dos obstáculos encontrados - o grupo pensa que o seu trabalho foi positivo. O produto final é funcional, satisfaz todos os Use Cases feitos e, acima de tudo, é flexível e utilizável o que sugere que trará bastante entusiasmo aos futuros utilizadores da aplicação.
