Ao nível do desenvolvimento da aplicação foi necessário dispor de recursos humanos, de hardware e de software.
Os recursos humanos englobam os trabalhadores envolvidos no projeto e as horas dedicadas neste mesmo. Quanto aos recursos de hardware indispensáveis para o desenvolvimento da “PORTOURGAL” foram as máquinas pessoais dos elementos do grupo.
Para além destes aspetos, foram utilizadas as seguintes ferramentas no decorrer do desenvolvimento do projeto:
\begin{itemize}
    \item \textbf{Visual Paradigm} – plataforma usada para a modelação do projeto;
    \item \textbf{Latex} - plataforma usada para a escrita do relatório;
    \item \textbf{MongoDB Atlas} - plataforma usada para a persistência de dados;
    \item \textbf{Microsoft Visual Studio} - ambiente de desenvolvimento das aplicações .NET;
    \item \textbf{Microsoft .NET C#} - plataforma usada para a codificação da aplicação;
    \item \textbf{Adobex XD} - ferramenta de design para a construção dos mockups;
    \item \textbf{Postman} - plataforma usada para facilitar as chamadas à API;
    \item \textbf{Weather API} - API para obter o clima;
    \item \textbf{Microsoft Azure} - plataforma usada para dar host à API;
    \item \textbf{Google & Booking} - usado para o povoamento da Base de Dados.
\end{itemize}