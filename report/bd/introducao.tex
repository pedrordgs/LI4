Os sistemas de bases de dados relacionais assumem uma posição de predominância no mercado. Tal deve-se principalmente à eficácia do armazenamento dos dados estruturados. É importante referir que este tipo de sistemas preservam a consistência, integridade, isolamento dos dados e durabilidade. 
    
Contudo, surgiram problemas que conduziram ao aparecimento de sistemas de bases de dados não relacionais (NoSQL) que foram idealizados atendendo às lacunas que as bases de dados tradicionais demonstraram, com alta performance e capacidade de expansão. Estas, em vez de armazenarem os dados em tabelas, utilizam estruturas dependendo do tipo da base de dados, podendo estas serem, grafos, colunas, chave-valor e documentos.